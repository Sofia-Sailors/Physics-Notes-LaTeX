\documentclass{article}
\usepackage{amsmath}
\usepackage{gensymb}
\begin{document}
\section{Two-Dimentional Kinematics}
Many things move along curved paths. These curved paths are observed in two dimensions in kinematics. Blah Blah, I hate myself so much haha

\subsection{Introduction}
\subsubsection{Two-Dimensional Motion: Walking in a City}
If you walk from one point to another in a city with uniform square blocks, you can break down your travel by how many block you walked along each axis, with the direction included. Ex: 9 blocks East then 5 north. The straight-line path would be founf with the Pythagorean theorem: $a^2 + b^2 = c^2$, with $c$ as the straight-line distance.

\[c = \sqrt{a^2 + b^2} \]

$\sqrt{(9 blocks)^2 + (5 blocks)^2} = 10.3 blocks$, much shorter than the combined 14 block you would have to travel otherwise.

The straight-line distance being shorter than the total walked distance is one of the general characteristics of vectors

When working with one-dimensional kinematics, only one arrow is used. In two-dimensional Kinematics, we can use up to three: The final straight-line path, the vertical component, and the horizontal component. The Horizontal and vertical components(vectors) add up to make the straigh-line path.

\subsubsection{The Independence of Perpendicular Motions}
Each Direction is affected only by the motion in that direction.
Ex.: If 2 balls fall from a table, one dropped off (no horizontal motion) and one rolled off(Horizontal Motion), the horizontal motion will not be affected by the vertical motion. (remember the lab photo).

The two-dimensional  surved path of a thrown object is called \emph{projectile motion}. 

\subsection{Vector Addition and Subtraction: Graphical Methods}
\subsubsection{Vectors in Two Dimensions}
Vector - quantity that has a magnitude and direction.
Displacement, Velocity, acceleration, and force are all vectors.
When working with one-dimensional vectors, direction can be given with a plus or minus sign.
In two-dimensional vectors, we specify direction in relation to a reference frame.

\subsubsection{Vector: Head-to-Tail Method}
The head-to-tail method is a graphical way to add vectors (used in lab[thanks Nicole])*. The tail is the starting point and the head is the final point (textbook figure 3.10).

\begin{enumerate}
\item Draw an arrow that represents the first vector
\item Then, draw a second arrow starting representing the second vector, starting from the head of the first.
\item \emph{if there are more than two vectors, continue adding}, otherwise move to step 4
\item Draw an arrow from the tail of first vector (or origin) to the head of the last vector. This is the \textbf{resultant} of the other vectors.
\item Magnitude can be measured with either a ruler, or the Pythagorean theorem.
\item To get the direction, measure the angle it makes with the reference frame (protractor).  
\end{enumerate}

The numerical accuracy is determined by the accuracy of the drawing.

\subsubsection{Vector Subtraction}
Vector subtraction is the same as vector addition, but with (some) vegative vectors. Ex. Subtracting vector $B$ from vector $A$, or $A-B$, would be adding the \emph{negative} of vector $B$ ($A + (-B)$. Graphically, vector $-B$ would have the same magnitude of vector $B$, just with an opposite direction (equal but opposite).

\subsubsection{Multiplication of Vectors and Scalars}
Multiplying a vector by a positive scalar only changes the \textbf{magnitude} of the vector, not the direction.

Multiplying a vector by a \textbf{negative} scalar changes the direction(making it opposite), and the magnitude (if the scalar $\neq 0$).

\subsubsection{Resolving a Vector into Components}
Depending on circumstances, components must be determined from a single vector. Components are often expressed as $x-$ and $y-$ or north-south and east-west.
%Ex.: If we know the displacement and angle, then we can find the components rather simply, using Trigonometry.
%finding the $\sin$ of the angle can help us find the vertical component, and finding the $\cos$ of the angle can help us find the horizontal component. 

\subsection{Vector Addition and Subtraction: Analytical Methods}
\subsubsection{Resolving a Vector into Perpendicular Components}

For a vector $A$, it is made of 2 perpendicular vectors $A_x$ and $A_y$.
$A_x$ and $A_y$ are components of $A$ along x- and y-axes. Expressing this relationship as an equation, you get \[A_x + A_y = A\].
It is important to note that this addition applies for both magnitude\emph{and} direction, not just magnitude.

We can easily find the vector components by using some simple trigonometry:\[A_x = A\cos \theta\] and \[A_y = A\cos \theta\]

\subsubsection{Calculating a Resultant Vector}

If $A_x$ and $A_y$ are known, then vector $A$ can be found analytically. \[A = \sqrt{A_x^2 + A_y^2}\] \[\theta = \arctan (A_y/A_x)\]
This equation finding $A$ is just the pathagorean theorem.

\subsubsection{Adding Vectors Using Analytical Methods}
Figure 3.29 in the textbook
\begin{enumerate}
\item Identify x- andy -axes that will be used in the problem. then find the components of each vector to be added along the chosen perpendiculat axis. Use the previous equations for $A_x$ and $A_y$ to find the components.
\item Find the components of the resultant along each axis by adding the components of the individual vectore along that axis \[R_x = A_x + B_x\] \[R_y = A_y + B_y\]
\item To get the magnitude $R$ of the resultant, use the Pythagorean theorem: \[R = \sqrt{R_x^2 + R_y^2}\]
\item To get the direction of the resultant relative to x-axis: \[\theta = \arctan (R_y/R_x)\]
\end{enumerate}
\subsection{Projectile Motion}
\textbf{Projectile Motion} is the motion of an object thrown or projected into the air, subject to only the acceleration of gravity.
The object is called the projectile, and its path is called the trajectory.

\textbf{Important}, the motions along perpendicular axes are independent. This means that an object's horizontal motion won't affect its vertical motion, and vice versa. 

When dealing with falling/propelled objects, and air resistence is negligable, acceleration ($a_y$) is the value of gravity: $g = 9.80$.
The direction of acceleration depends on what is being measured; if a ball is thrown into the air, we recognize gravity as being negative. If a ball is dropped from a tall building, we might interpret it as positive (for calculation speed or smthn idk).

\subsubsection{Analyzing Projectile Motion}
\begin{enumerate}
\item Resolve or break the motion into horizontal and vertical components along x- and y-axes. Magnitudes of the somponents are found with $A_x = A\cos \theta$ $A_y = A\sin \theta$.
  
\item Treat the motion as two independent one-dimensional motions, one horizontal and one vertical. The book represents the Kinematic equations relative to vert. and horiz. motion.
  
\item Solve for the unknowns in the two seperate motions - one horizontal and on vertical. The \emph{only} common variable between them is \textbf{time}($t$). Treat as if it were one-dimensional kinematics.
  
\item Recombine the two motions to find the total displacement and velocity. Reference book once again. Revise before test.
\end{enumerate}

For maximum height $y=h$: \[h = \frac{v^2_{0y}}{2g}\]

For range $R$ of a projectile on level ground with negligible air resistance: \[R = \frac{v_0^2 \sin 2 \theta _0}{g}\]
with $v_0$ as initial speed and $\theta _0$ as the initial angle (relative to horizon)

\subsection{Addition of Velocities}
\subsubsection{Relative Velocity}
If a person rows a boat across a moving river heading to shore, they will move diagonally relative to shore. The object has a velocity relative to a medium(river) and that medium has a velocity relative to an observer on solid ground. The object's velocity \emph{relative to the \textbf{observer}} is the sum of these velocity vectors.

Adding Velocities:
In one-dimensional motion, adding velocities is as simple as adding the magnitudes.
In two-dimensional motion, you can use either graphical or analytical techniques to ass velocities.
Using analytical techniques we can find relationships b/w magnitude and direction of velocity ($v$ and $\theta$) and it's components ($v_x$ and $v_y$) along the x- and y-axes:
\[ v_x = v \cos \theta\] \[v_y = v \sin \theta\] \[ v = \sqrt{v_x^2 + v_y^2} \] \[\theta = \arctan (v_y / v_x) \]
$v_x$: horizontal component, $v_y$: vertical component, $v$: hypotenuse, $\theta$: angle.

\subsubsection{Relative Velocities and Classial Relativity}
When adding velocities, it is important to mention that they are relative to a reference frame. These are \textbf{relative velocities}.

\textbf{Classical relativity} - more Galileo and Newton, less Einstein. Used when speeds are less than 1 percent the speed of light.
\end{document}
