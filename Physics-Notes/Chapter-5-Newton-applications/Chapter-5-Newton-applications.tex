\documentclass{article}
\usepackage{amsmath}
\begin{document}
\section{Friction}

\textbf{Friction} is the force that opposes relative motion between surfaces in contact. 
It is parallel to the contact surface between surfaces and always in the direction \emph{opposite} motion or attempted motion of systems relative to eachother.

\[F_f = \mu F_N \]

\textbf{Kinetic Friction} - If two surfaces are in contact and moving relative to one another.

\textbf{Static Friction} - If two surfaces are in contact and \emph{\textbf{NOT}} moving relative to one another. Static friction is usually greator than kinetic friction between the surfaces.

The magnitude of Static friction $f_s$ is \[ f_s \leq \mu _s N \]

with $ \mu _s $ is the coefficient of static friction and $N$ being the magnitide of normal force. Also represented as $ F_f \leq  \mu _s F_N$

once the applied force exceeds $f_{s(max)}$ then the object moves, giving us \[F_{s(max)} = \mu _s F_N \]

Once moving, the magnitude of Kinetic friction $F_k$ is given

\[F_k = \mu _k F_N\]


\section{Drag}

Drag force always moves in the opposite direction of an objects motion (like friction)

Drag force $F_D$ is proportional to the square of the speed of an object.
$F_D \propto v^2$

\[ F_D = \frac{1}{2}C \rho Av^2 \]

With $C$ as the drag coefficiant, A being the area of the object facing the fluid and $\rho$ being the density of the fluid. More generalized equation is $F_D = bv^2$ where $b$ is a constant equivalent to $0.5C \rho A$.

EX: At Terminal Velocity:
\[ F_{net} = mg - F_D = ma = 0\]
so \[mg - F_D = 0 ---> mg = F_D \]
using the equation for drag force, we get: \[ mg = \frac{1}{2} \rho C A v^2 \]
solving for velocity:
\[v = \sqrt{ \frac{2mg}{\rho C A}}\]


\textbf{Stoke's Law}: $F_s = 6 \pi r \eta v $
Where $r$ is the radius of the object, $\eta$ is the viscosity of the fluid, and $v$ s velocity


\section{Elasticity}
\textbf{Hooke's Law}: \[F = k \Delta L \]
with $\Delta L$ being the amount of deformation (like change in length) produced by force $F$, and $k$ is a proportionality constant.
\[ \Delta L = \frac {F}{k} \]




\end{document}
