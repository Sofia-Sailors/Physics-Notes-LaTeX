\documentclass{article}
\usepackage{amsmath}
\usepackage{gensymb}
\begin{document}
\section{Two-Dimentional Kinematics}
Many things move along curved paths. These curved paths are observed in two dimensions in kinematics. Blah Blah, I hate myself so much haha

\subsection{Introduction}
\subsubsection{Two-Dimensional Motion: Walking in a City}
If you walk from one point to another in a city with uniform square blocks, you can break down your travel by how many block you walked along each axis, with the direction included. Ex: 9 blocks East then 5 north. The straight-line path would be founf with the Pythagorean theorem: $a^2 + b^2 = c^2$, with $c$ as the straight-line distance.

\[c = \sqrt{a^2 + b^2} \]

$\sqrt{(9 blocks)^2 + (5 blocks)^2} = 10.3 blocks$, much shorter than the combined 14 block you would have to travel otherwise.

The straight-line distance being shorter than the total walked distance is one of the general characteristics of vectors

When working with one-dimensional kinematics, only one arrow is used. In two-dimensional Kinematics, we can use up to three: The final straight-line path, the vertical component, and the horizontal component. The Horizontal and vertical components(vectors) add up to make the straigh-line path.

\subsubsection{The Independence of Perpendicular Motions}
Each Direction is affected only by the motion in that direction.
Ex.: If 2 balls fall from a table, one dropped off (no horizontal motion) and one rolled off(Horizontal Motion), the horizontal motion will not be affected by the vertical motion. (remember the lab photo).

The two-dimensional  surved path of a thrown object is called \emph{projectile motion}. 

\subsection{Vector Addition and Subtraction: Graphical Methods}
\subsubsection{Vectors in Two Dimensions}
Vector - quantity that has a magnitude and direction.
Displacement, Velocity, acceleration, and force are all vectors.
When working with one-dimensional vectors, direction can be given with a plus or minus sign.
In two-dimensional vectors, we specify direction in relation to a reference frame.

\subsubsection{Vector: Head-to-Tail Method}
The head-to-tail method is a graphical way to add vectors (used in lab[thanks Nicole])*. The tail is the starting point and the head is the final point (textbook figure 3.10).

\begin{enumerated}
\item Draw an arrow that represents the first vector
\item Then, draw a second arrow starting representing the second vector, starting from the head of the first.
\item \emph{if there are more than two vectors, continue adding}, otherwise move to step 4
\item Draw an arrow from the tail of first vector (or origin) to the head of the last vector. This is the \textbf{resultant} of the other vectors.
\item Magnitude can be measured with either a ruler, or the Pythagorean theorem.
\item To get the direction, measure the angle it makes with the reference frame (protractor).  
\end{enumerate}

The numerical accuracy is determined by the accuracy of the drawing.

\subsubsection{Vector Subtraction}
Vector subtraction is the same as vector addition, but with (some) vegative vectors. Ex. Subtracting vector $B$ from vector $A$, or $A-B$, would be adding the \emph{negative} of vector $B$ ($A + (-B)$. Graphically, vector $-B$ would have the same magnitude of vector $B$, just with an opposite direction (equal but opposite).

\subsubsection{Multiplication of Vectors and Scalars}
Multiplying a vector by a positive scalar only changes the \textbf{magnitude} of the vector, not the direction.

Multiplying a vector by a \textbf{negative} scalar changes the direction(making it opposite), and the magnitude (if the scalar $\neq 0$).

\subsubsection{Resolving a Vector into Components}
Depending on circumstances, components must be determined from a single vector. Components are often expressed as $x-$ and $y-$ or north-south and east-west.
Ex.: If we know the displacement and angle, then we can find the components rather simply, using Trigonometry.
finding the $\sin$ of the angle can help us find the vertical component, and finding the $\cos$ of the angle can help us find the horizontal component. 

\subsection{Vector Addition and Subtraction: Analytical Methods}


\subsection{Projectile Motion}
\subsection{Addition of Velocities}
























\end{document}
