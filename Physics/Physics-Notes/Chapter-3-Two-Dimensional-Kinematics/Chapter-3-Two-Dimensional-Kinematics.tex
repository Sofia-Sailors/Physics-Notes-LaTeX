\documentclass{article}
\usepackage{amsmath}
\usepackage{gensymb}
\begin{document}
\section{Two-Dimentional Kinematics}
Many things move along curved paths. These curved paths are observed in two dimensions in kinematics. Blah Blah, I hate myself so much haha

\subsection{Introduction}
\subsubsection{Two-Dimensional Motion: Walking in a City}
If you walk from one point to another in a city with uniform square blocks, you can break down your travel by how many block you walked along each axis, with the direction included. Ex: 9 blocks East then 5 north. The straight-line path would be founf with the Pythagorean theorem: $a^2 + b^2 = c^2$, with $c$ as the straight-line distance.

\[c = \sqrt{a^2 + b^2} \]

$\sqrt{(9 blocks)^2 + (5 blocks)^2} = 10.3 blocks$, much shorter than the combined 14 block you would have to travel otherwise.

The straight-line distance being shorter than the total walked distance is one of the general characteristics of vectors

When working with one-dimensional kinematics, only one arrow is used. In two-dimensional Kinematics, we can use up to three: The final straight-line path, the vertical component, and the horizontal component. The Horizontal and vertical components(vectors) add up to make the straigh-line path.

\subsubsection{The Independence of Perpendicular Motions}
Each Direction is affected only by the motion in that direction.
Ex.: If 2 balls fall from a table, one dropped off (no horizontal motion) and one rolled off(Horizontal Motion), the horizontal motion will not be affected by the vertical motion. (remember the lab photo).

The two-dimensional  surved path of a thrown object is called \emph{projectile motion}. 

\subsection{Vector Addition and Subtraction: Graphical Methods}
\subsubsection{Vectors in Two Dimensions}
Vector - quantity that has a magnitude and direction.
Displacement, Velocity, acceleration, and force are all vectors.
When working with one-dimensional vectors, direction can be given with a plus or minus sign.
In two-dimensional vectors, we specify direction in relation to a reference frame.

\subsection{Vector Addition and Subtraction: Analytical Methods}
\subsection{Projectile Motion}
\subsection{Addition of Velocities}
























\end{document}
