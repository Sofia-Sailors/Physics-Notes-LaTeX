\documentclass{article}
\usepackage{amsmath}
\usepackage{gensymb}
\begin{document}
\section{Kinematics}
\subsection{Displacement}
\subsubsection{Position}
Position - Where an object is at any given time relative to a reference frame.
Often, Earth is used as a reference frame, especially when referring to an objects position \emph{on earth}, such as coordinates and roads.
when rreferring to an object's location with in a container, such as a house, bus, or plane, you would use that container as the reference frame.
\subsubsection{Displacement}
Displacement - the change in position of an object relative to its reference frame. \textbf{IT IS NOT DISTANCE!!!}
Displacement is found with $\Delta{x} = x_f - x_0$, where $\Delta{x}$ is displacement, $x_f$ is final position and $x_0$ is the initial position.
The Greek character $\Delta$ means ``change in'' a quantity, so $\Delta{t}$ would represent a change in time, or $\Delta{x}$ would represent a change in distance, as the previous equation shows.
Displacement has Direction and Magnitude, so if someone moved to a seat to the right their direction would be ``right'' and their magnitude would be ``1 seat''.
\subsubsection{Distance}
Distance is the magnitude \textbf{\emph{or}} displacement between two positions. Direction is not applicable.
Important to note: Distance between two points and Distance traveled are \emph{different things}. Distance Traveled is the \emph{total length of the path traveled between two positions.}
Distance has no direction, so there is no applicable sign.

\subsection{Vectors, Scalars, and Coordinate Systems}
\subsubsection{Vector}
A \textbf{vector} is any quantity with both direction and magnitude
Ex.: Velocity of 90km/h east and a force of 500N down
The Direction of a Vector in 1-dimensional motion is given by a (+) or (-). Graphically, vectors are represented by arrows. an arrow used to represent magnitude has a length proportional to its magnitude and points in the same direction.

\subsubsection{Scalar}

Some quantities(distance) have either no, or an unspecified direction.
A \textbf{scalar} is any quantity that has a magnitude, but no direction. Ex.: 25 m/h, height.
Scalars are never represented by arrows.

\subsubsection{Coordinate Systems}
In order to describe the direction of a vector quantity, you must establish a coordinate system within the reference frame.
For one-dimensional motion, there is just a one-dimensional coordinate line
For horizontal motion, motion to the right is usually positive, and motion to the left is usually considered negative.
For Vertical motion, up is usually positive and down is usually negative.
Sometimes, depending on circumstance, it is better to switch positions, such as finding distance fallen for a falling object.
The positive direction can not be changed once you start solving for a problem.

\subsection{Time, Velocity, and Speed}

\subsubsection{Time}

Every measurment of time involves measuring a change in some physical quantity.
In physics, \emph{time is change}.

\subsubsection{Velocity}
Amount of displacement divided by the elapsed time ($\Delta{t}$),

\[\bar{v} = \frac{\Delta{x}}{\Delta{t}} = \frac{x_f - x_0}{t_f - t_0},\]
where $\bar{v}$ is the average velocity, $\Delta{x}$ is the change in position (also known as displacement), and $x_f$ and $x_0$ are the final and beginning positions at time $t_f$ and $t_0$.
If the initial time ($t_0$) is 0, then the equation can be represented as \[\bar{v} = \frac{\Delta{x}}{t}\]

This Definition indicates that \emph{velocity} is a vector because \emph{displacement} is a vector(contains both magnitude and direction). Unit is m/s, but other units may be used in different circumstances (so check the question).
Ex.: If an airplane passenger took 5 seconds(s) to move -4m(towards the back of the plane), then their movement could be represented as \[\bar{v} = \frac{\Delta{x}}{\Delta{t}} = \frac{-4 m}{5 s} = -0.8 m/s\]
Average velocity only tells us the big picture, not the smaller inconsistencies. For more detail, measurements must be taken from smaller segments over smaller time intervals.
Using smaller measurements, we can find the \textbf{Instantaneous velocity}

\subsubsection{Speed}
Speed and Velocity are two different things. \textbf{Speed} has no direction, and is \emph{scalar}.
\textbf{Instantaneous Speed} is the \emph{magnitude} of instantaneous velocity.

Ex.: If the airplane passenger had an instantaneous velocity of -3.0 m/s, then their instantaneous \emph{speed} is 3.0 m/s (direction(-) was not used/considered).

Ex.2: If you were traveling at an instantaneous velocity of 40km/h due north, then your Instantaneous speed would be 40km/h (no direction applicable).

\textbf{Average speed} is different, instead being the distance divided by time (like velocity), without care for direction.

Just as distance can be greater than the magnitude of displacement, average speed can be greater than average velocity, as velocity measures displacement over time.

Ex. If you travel to the store and back in 30 mins, with the total distance traveled being 6km, then your average speed ($\frac{x_f}{\Delta{t}}$) was 12km/h, but your average velocity ($\frac{\Delta{x}}{\Delta{t}}$) is 0, as the total displacement is 0

This shows that \textbf{average speed} is \emph{not} just the magnitude of average velocity

The motion of an object can also be represented with a graph, but I don't have the energy to deal with a graph right now. Whoever reads this next, \textbf{\emph{PLEASE}} tell me to fix it.

\subsection{Acceleration}
\subsubsection{Acceleration}
Acceleration is the rate of change of velocity, or \textbf{the rate of change of the rate of change}.
The greator the acceleration, the greater the change in velocity over a given time.
\textbf{Average Acceleration} is given with:

\[\bar{a} = \frac{\Delta{v}}{\Delta{t}} = \frac{v_f - v_0}{t_f - t_0},\]

where $\bar{a}$ is acceleration, $v$ is velocity and $t$ is time.

Because accelertion is Velocity divided by time in (s), the units for acceleration are $m/s^2$, meaning how many meters per second the velocity changes per second.

- - \textbf{Acceleration as a Vector} - -

Acceleration is a vector in the same direction as the change in velocity, $\Delta{v}$. As a vector, it can change in either speed, direction, or both.

Deceleration is when accelerations direction opposes the change in motion

Deceleration is \textbf{NOT} ``negative acceleration''. Negative acceleration is acceleration in the negative direction (in the chosen coordinate system)

Ex.: If a car is going foward and applies the brakes, they are decelerating. If the car then begins reversing (and the coordinate system stays the same) then they have \textbf{negative acceleration}.

\subsubsection{Instantaneous Acceleration}
Instantaneous Acceleration is the acceleration at a specific instance in time. It is found by the same process used to find instantaneous velocity; measure the change in velosity over an infinitesimally small period of time.

Ex.: If a car's velocity changes from 45m/s to 46m/s over an interval of 0.1 seconds, then the car is accelerating at a rate of 10km/h at that instance

Units are (maybe)wrong for this:
\[a = \frac{\Delta{v}}{\Delta{t}} = \frac{(46-45)m/s}{0.1} = 1\]


\subsection{Motion Equations for Constant Acceleration in One Direction}

\subsubsection{Notation: \emph{t, x, v, a}}

When writing equations, notation is important. Typically, in terms of measured units, the subscript 0 is used to represent initial {value}, and the final value does not have a subscript at all. So, instead of $v_i$ and $v_f$, it would be represented by $v_0$ and $v$.

\[\left. \begin{array}{l}
  \Delta{t} = t \\ \Delta{x} = x - x_0 \\ \Delta{v} = v - v_0 \end{array} \right\} \]

For the sake of this chapter's notes, we are going to assume that $a = k$ or $\bar{a} = a = constant$.


\subsubsection{Solving for displacement ($\Delta(x)$) and final position ($x$) from average velocity when $a = k$}
  
Average Velocity \[\bar{v} = \frac{\Delta{x}}{\Delta{t}}\]

then substituting for $\Delta{x}$ and $\Delta{t}$:

\[\bar{v} = \frac{x - x_0}{t}\]

Now, solving for $x$ (by multiplying and moving):

\[ x = x_0 + \bar{v} t \],

where average velocity is:

\[\bar{v} = \frac{v_0 + v}{2}\](constant $a$)

When $a=k$, then velocity is just the average of initial and final velocities.

\subsubsection{Displacement}

Displacement is given by $x = x_0 + \bar{v}t$. Useful when trying to find final distance or time, if $\bar{v}$ is provided.
This equation shows that, when a = constant, displacement is a linear function of $\bar{v}$.

\subsubsection{Solving for Final Velocity}

\[ a = \frac{\Delta{v}}{\Delta{t}}\]

substituting for $\Delta{v}$ and $\Delta{t}$:

$ a = \frac{v - v_0}{t} $(constant $a$)

Then solving for $v$:

$v = v_0 + at$(constant $a$)

This equation provides insight into the relationships at play
\begin{itemize}
\item final velocity depends on acceleration size and duration
\item if $a=0$ then $v = v_0$, making velocity constant
\item if $a$ is negative, then $v$ will be \emph{less} than $v_0$
\end{itemize}

\subsubsection{Solving for Final Velocity ($ a \neq  0$)}

When Velocity is changing (i.e. acceleration is not 0), we have to combine our different Final velocity equations:
\[ v = v_0 + at \]

then we add $v_0$ to both(?) sides and divide by 2 (I dont know exactly how it ended up like this; blame textbook):

\[\frac{v_0 + v}{2} = v_0 + \frac{1}{2}at.\]

We know that $\frac{v_0 + v}{2}$ is equal to our average velocity $\bar{v}$, so:

\[ \bar{v} = v_0 + \frac{1}{2}at\]

Now, we substitute this equation for $\bar{v}$ into our displacement equation, $x = x_0 + \bar{v}t$:

$ x = x_0 + (v_0 + \frac{1}{2}at)t$

$ x = x_0 + v_0t + \frac{1}{2}at^2$ (constant a)

\begin{itemize}
\item When acceleration is not 0, displacement depends on the square of the elapsed time
\item if acceleration is 0, then $v_0 = \bar{v}$ and $ x = x_0 + v_0 t + \frac{1}{2}at^2$ becomes $ x = x_0 + v_0 t$
\end{itemize}

\subsubsection{Timeless}

We can also use the equation $v = v_0 + at$ to find t:
\[t = \frac{v - v_0}{a}\]

Now, we can substitute this value of $t$ with the value for $\bar{v}$, $\bar{v} = \frac{v_0 + v}{2}$ into the equation $x = x_0 + \bar{v}t$ :

$x = x_0 + \bar{v}t$

$x = x_0 + (\frac{v_0 + v}{2}) * (\frac{v - v_0}{a})$

$x = x_0 + \frac{v^2 + v_0^2}{2a}$

$\frac{v^2 + v_0^2}{2a} = x - x_0$

$v^2 - v_0^2 = 2a(x - x_0)$

$v^2 = v_0^2+ 2a(x - x_0)$ (constant a).

 Notice how this equation lacks \emph{time} as a variable; it was factored out with $t = \frac{v_0 + v}{a}$. That is why this equation is known as the \textbf{Timeless Equation}.

 \subsubsection{Equation compilation}

Final Distance: ($a=k$) $x = x_0 + \bar{v}t$

 Average Velocity: ($a=k$) $\bar{v} = \frac{\bar{v} + v}{2}$

Final Velocity: ($a\neq k$) $v = v_0 + at$

Final Distance: ($a \neq k$) $x = x_0 + v_0 t + \frac{1}{2}at^2$

Timeless Equation: $v^2 = v_0^2 + 2a(x - x_0)$ 
 
\subsection{Problem-Solving Basics for One-Dimension Kinematics}

\subsubsection{Step 1}
Examine the situation to determine physical principles involved. Draw a sketch if needed. Decide the values of the directions (pos, neg). understand the conceptual problem rather than just numbers.
\subsubsection{Step 2}
Make a list of what is given or can be inferred. Find the knowns. If something is ``stopped'', it means zero, and often times the initial time and position are zero.
\subsubsection{Step 3}
Idenify what needs to be determined in the problem (the unknowns).
\subsubsection{Step 4}
Determine the equation or set of equations that will be most helpful. Use the list of knowns and unknowns. If time is given, don't use the timeless equation. try and find equations with just one unknown, then solve for it. Sometimes there are multiple unknowns.
\subsubsection{Step 5}
Substitute the knowns \emph{along with their units} into the applicable equation(s), then obtain numerical solutions \emph{with} units. Don't forget the significant figures.
\subsubsection{Step 6}
Determine if the answer is reasonable. a shopping cart going >100km/h isn't particularly reasonable. Check both Magnitude \emph{and} sign. ensure it has the proper direction.

\subsubsection{Unreasonable Results}
The Book has some stuff about how figuring out reasonable vs. unreasonable results improves your skills. I don't particularly care that much.




\subsection{Falling Objects}

Falling objects are in an independent class of motion problems.

\subsubsection{Gravity}
Gravity's acceleration is consistent, at about $9.80 m/s^2$, often represented by $g$.

The direction of $g$ is \emph{downwards}. The positive or negative value of $g/a$ depends on how we define our coordinate system. If up is Positive, then $a=-g=-9.80m/s^2$. If Down is positive, then $a=g=9.80m/s^2$

\subsubsection{One-Dimensional Motion involving Gravity}
Considering straight up and down motion with no air resistance or friction. This would mean that velocity is vertical. If an object is dropped, its initial velocity($v_0$) is 0. once the object is in free-fall, the motion is one-directional w/a constant acceleration of magnitude $g$. (for free-fall, up is positive and down is negative[in this circumstance], therefor acceleration is -g)

\[ v = v_0 - gt\]

\[y = y_0 + v_0 t - \frac{1}{2}gt^2\]

\[v^2 = v_0^2 - 2g(y - y_0)\]

\subsection{Graphical Analysis od One-dimensional Motion}


  
\end{document}
